%%%%%%%%%%%%%%%%%%%%%%%%%%%%%%%%%%%%%%%%%%%%%%%%%%%%%%%%%%%%%%%
%% BRIEF VERSION OF OXFORD THESIS TEMPLATE FOR CHAPTER PREVIEWS

%%%%% CHOOSE PAGE LAYOUT
% format for PDF output (ie equal margins, no extra blank pages):
\documentclass[a4paper,nobind]{templates/ociamthesis}

% UL 5 January 2021 - add packages used by kableExtra
\usepackage{booktabs}
\usepackage{longtable}
\usepackage{array}
\usepackage{multirow}
\usepackage{wrapfig}
\usepackage{colortbl}
\usepackage{pdflscape}
\usepackage{tabu}
\usepackage{threeparttable}
\usepackage{threeparttablex}
\usepackage[normalem]{ulem}
\usepackage{makecell}
\usepackage[colorlinks=false,pdfpagelabels,hidelinks=]{hyperref}
\usepackage{float}


%UL set section header spacing
\usepackage{titlesec}
% 
\titlespacing\subsubsection{0pt}{24pt plus 4pt minus 2pt}{0pt plus 2pt minus 2pt}

% UL 30 Nov 2018 pandoc puts lists in 'tightlist' command when no space between bullet points in Rmd file
\providecommand{\tightlist}{%
  \setlength{\itemsep}{0pt}\setlength{\parskip}{0pt}}
 
% UL 1 Dec 2018, fix to include code in shaded environments
\usepackage{color}
\usepackage{fancyvrb}
\newcommand{\VerbBar}{|}
\newcommand{\VERB}{\Verb[commandchars=\\\{\}]}
\DefineVerbatimEnvironment{Highlighting}{Verbatim}{commandchars=\\\{\}}
% Add ',fontsize=\small' for more characters per line
\usepackage{framed}
\definecolor{shadecolor}{RGB}{248,248,248}
\newenvironment{Shaded}{\begin{snugshade}}{\end{snugshade}}
\newcommand{\AlertTok}[1]{\textcolor[rgb]{0.94,0.16,0.16}{#1}}
\newcommand{\AnnotationTok}[1]{\textcolor[rgb]{0.56,0.35,0.01}{\textbf{\textit{#1}}}}
\newcommand{\AttributeTok}[1]{\textcolor[rgb]{0.77,0.63,0.00}{#1}}
\newcommand{\BaseNTok}[1]{\textcolor[rgb]{0.00,0.00,0.81}{#1}}
\newcommand{\BuiltInTok}[1]{#1}
\newcommand{\CharTok}[1]{\textcolor[rgb]{0.31,0.60,0.02}{#1}}
\newcommand{\CommentTok}[1]{\textcolor[rgb]{0.56,0.35,0.01}{\textit{#1}}}
\newcommand{\CommentVarTok}[1]{\textcolor[rgb]{0.56,0.35,0.01}{\textbf{\textit{#1}}}}
\newcommand{\ConstantTok}[1]{\textcolor[rgb]{0.00,0.00,0.00}{#1}}
\newcommand{\ControlFlowTok}[1]{\textcolor[rgb]{0.13,0.29,0.53}{\textbf{#1}}}
\newcommand{\DataTypeTok}[1]{\textcolor[rgb]{0.13,0.29,0.53}{#1}}
\newcommand{\DecValTok}[1]{\textcolor[rgb]{0.00,0.00,0.81}{#1}}
\newcommand{\DocumentationTok}[1]{\textcolor[rgb]{0.56,0.35,0.01}{\textbf{\textit{#1}}}}
\newcommand{\ErrorTok}[1]{\textcolor[rgb]{0.64,0.00,0.00}{\textbf{#1}}}
\newcommand{\ExtensionTok}[1]{#1}
\newcommand{\FloatTok}[1]{\textcolor[rgb]{0.00,0.00,0.81}{#1}}
\newcommand{\FunctionTok}[1]{\textcolor[rgb]{0.00,0.00,0.00}{#1}}
\newcommand{\ImportTok}[1]{#1}
\newcommand{\InformationTok}[1]{\textcolor[rgb]{0.56,0.35,0.01}{\textbf{\textit{#1}}}}
\newcommand{\KeywordTok}[1]{\textcolor[rgb]{0.13,0.29,0.53}{\textbf{#1}}}
\newcommand{\NormalTok}[1]{#1}
\newcommand{\OperatorTok}[1]{\textcolor[rgb]{0.81,0.36,0.00}{\textbf{#1}}}
\newcommand{\OtherTok}[1]{\textcolor[rgb]{0.56,0.35,0.01}{#1}}
\newcommand{\PreprocessorTok}[1]{\textcolor[rgb]{0.56,0.35,0.01}{\textit{#1}}}
\newcommand{\RegionMarkerTok}[1]{#1}
\newcommand{\SpecialCharTok}[1]{\textcolor[rgb]{0.00,0.00,0.00}{#1}}
\newcommand{\SpecialStringTok}[1]{\textcolor[rgb]{0.31,0.60,0.02}{#1}}
\newcommand{\StringTok}[1]{\textcolor[rgb]{0.31,0.60,0.02}{#1}}
\newcommand{\VariableTok}[1]{\textcolor[rgb]{0.00,0.00,0.00}{#1}}
\newcommand{\VerbatimStringTok}[1]{\textcolor[rgb]{0.31,0.60,0.02}{#1}}
\newcommand{\WarningTok}[1]{\textcolor[rgb]{0.56,0.35,0.01}{\textbf{\textit{#1}}}}

%UL 2 Dec 2018 add a bit of white space before and after code blocks
\renewenvironment{Shaded}
{
  \vspace{10pt}%
  \begin{snugshade}%
}{%
  \end{snugshade}%
  \vspace{8pt}%
}
%UL 2 Dec 2018 reduce whitespace around verbatim environments
\usepackage{etoolbox}
\makeatletter
\preto{\@verbatim}{\topsep=0pt \partopsep=0pt }
\makeatother

%UL 28 Mar 2019, enable strikethrough
\usepackage[normalem]{ulem}

%UL use soul package for correction highlighting
\usepackage{soul}
\usepackage{xcolor}
\newcommand{\ctext}[3][RGB]{%
  \begingroup
  \definecolor{hlcolor}{#1}{#2}\sethlcolor{hlcolor}%
  \hl{#3}%
  \endgroup
}
\soulregister\ref7
\soulregister\cite7
\soulregister\autocite7
\soulregister\textcite7
\soulregister\pageref7

%UL 3 Nov 2019, avoid mysterious error from not having hyperref included
\usepackage{hyperref}

%%%%% SELECT YOUR DRAFT OPTIONS
% Three options going on here; use in any combination.  But remember to turn the first two off before
% generating a PDF to send to the printer!

% This adds a "DRAFT" footer to every normal page.  (The first page of each chapter is not a "normal" page.)

% This highlights (in blue) corrections marked with (for words) \mccorrect{blah} or (for whole
% paragraphs) \begin{mccorrection} . . . \end{mccorrection}.  This can be useful for sending a PDF of
% your corrected thesis to your examiners for review.  Turn it off, and the blue disappears.

%%%%% BIBLIOGRAPHY SETUP
% Note that your bibliography will require some tweaking depending on your department, preferred format, etc.
% The options included below are just very basic "sciencey" and "humanitiesey" options to get started.
% If you've not used LaTeX before, I recommend reading a little about biblatex/biber and getting started with it.
% If you're already a LaTeX pro and are used to natbib or something, modify as necessary.
% Either way, you'll have to choose and configure an appropriate bibliography format...

% The science-type option: numerical in-text citation with references in order of appearance.
% \usepackage[style=numeric-comp, sorting=none, backend=biber, doi=false, isbn=false]{biblatex}
% \newcommand*{\bibtitle}{References}

% The humanities-type option: author-year in-text citation with an alphabetical works cited.
% \usepackage[style=authoryear, sorting=nyt, backend=biber, maxcitenames=2, useprefix, doi=false, isbn=false]{biblatex}
% \newcommand*{\bibtitle}{Works Cited}

%UL 3 Dec 2018: set this from YAML in index.Rmd
\usepackage[style=numeric-comp, sorting=none, backend=biber, doi=false, isbn=false]{biblatex}
\newcommand*{\bibtitle}{References}

% This makes the bibliography left-aligned (not 'justified') and slightly smaller font.
\renewcommand*{\bibfont}{\raggedright\small}

% Change this to the name of your .bib file (usually exported from a citation manager like Zotero or EndNote).
\addbibresource{references.bib}

%%%%% YOUR OWN PERSONAL MACROS
% This is a good place to dump your own LaTeX macros as they come up.

% To make text superscripts shortcuts
	\renewcommand{\th}{\textsuperscript{th}} % ex: I won 4\th place
	\newcommand{\nd}{\textsuperscript{nd}}
	\renewcommand{\st}{\textsuperscript{st}}
	\newcommand{\rd}{\textsuperscript{rd}}

%%%%% THE ACTUAL DOCUMENT STARTS HERE
\begin{document}

%%%%% CHOOSE YOUR LINE SPACING HERE
% This is the official option.  Use it for your submission copy and library copy:
\setlength{\textbaselineskip}{22pt plus2pt}
% This is closer spacing (about 1.5-spaced) that you might prefer for your personal copies:
%\setlength{\textbaselineskip}{18pt plus2pt minus1pt}

% UL: You can set the general paragraph spacing here - I've set it to 2pt (was 0) so
% it's less claustrophobic
\setlength{\parskip}{2pt plus 1pt}

% Leave this line alone; it gets things started for the real document.
\setlength{\baselineskip}{\textbaselineskip}

% all your chapters and appendices will appear here
\hypertarget{background}{%
\chapter{Background}\label{background}}

\adjustmtc
\markboth{Background}{}

\hypertarget{disease-surveillance-and-small-area-estimation}{%
\section{Disease surveillance and small-area estimation}\label{disease-surveillance-and-small-area-estimation}}

Small-area estimation methods aim to estimate population indicators for subgroups, typically in situations where direct estimates perform poorly due to data limitations.
These subgroups may often correspond to small geographic areas.
Small-area estimation methods have been used in a wide range of fields.
The Small-Area Health Statistics Unit (SASHU) at Imperial College London was set-up to monitor health around point sources of environmental pollution in response to the Sellafield enquiry into the increased incidence of childhood leukemia leukaemia near a nuclear reprocessing plant \autocite{elliott1992small}.
The research of SASHU has a focus on ratios of observed events to expected events, and testing hypothesis about hot-spots.

\hypertarget{hivaids}{%
\section{HIV/AIDS}\label{hivaids}}

\begin{Shaded}
\begin{Highlighting}[]
\NormalTok{plhiv2022 }\OtherTok{\textless{}{-}} \DecValTok{38000000}
\NormalTok{deaths2022 }\OtherTok{\textless{}{-}} \DecValTok{700000}
\NormalTok{infections2022 }\OtherTok{\textless{}{-}} \DecValTok{1700000}
\end{Highlighting}
\end{Shaded}

According to latest estimates, in 2022 \ensuremath{3.8\times 10^{7}} people are living with HIV, there were \ensuremath{7\times 10^{5}} AIDS-related deaths, and there were \ensuremath{1.7\times 10^{6}} people newly infected with HIV.
Surveillance is used is conducted to track epidemic trends, identify at-risk populations, find drivers of transmission, and evaluate the impact of prevention and treatment programs.
Sub-Saharan Africa is the most affected region.
Within sub-Saharan Africa, disease burden is unevenly distributed in space and across communities and individuals.
Key populations include men who have sex with men, female sex workers, people who inject drugs, transgender people, incarcerated people.
Larger demographic groups of higher risk include adolescent girls and young women.
Key HIV indicators are HIV prevalence, HIV incidence, coverage of ART and other interventions.
Key interventions are ART, condoms, PrEP and PEP, education, economic empowerment, VMMC.

There are significant data related difficulties including sparsity in space and time, survey bias, conflicting information sources, hard to reach populations, changing demographies.
These data limitations foreground the importance of synthesising multiple sources of information to obtain estimates.
Doing so increases the difficulty and complexity of the statistical modelling required.

Aims for HIV response going forward, and surveillance capabilities are needed to meet them.
Phasing out of nationally-representative household surveys for HIV.

Methods for prevention prioritisation include geographic, demographic, key population services, risk screening, individual-level risk characteristics.
Are there differences in effectiveness of treatments for different groups.

The population strategy \autocite{rose2001sick} is based on reducing risk factors across an entire population.
The individual strategy focuses on prevention in high-risk individuals.

\hypertarget{bayesian-spatio-temporal-statistics}{%
\section{Bayesian spatio-temporal statistics}\label{bayesian-spatio-temporal-statistics}}

Bayesian statistics is a statistical paradigm which, at its best, lets the analyst focus their attention on modelling the data at hand.
In particular, the primary concern is construction of a generative model for the observed data
\[
y \sim p(y).
\]

Given a generative model, computation of the posterior distribution \(p(y \, | \, \theta)\) proceeds using approximate Bayesian inference methods.
Markov chain Monte Carlo (MCMC) is the most popular approach, and proceeds by simulating samples from a Markov chain with stationary distribution equal to the distribution of interest.
Variational Bayes approaches assume the posterior distribution belongs to some class and use optimisation to choose the best member of that class.
Particular properties of spatio-temporal models make integrated nested Laplace approximations, if feasible, often the best option
Empirical Bayes approaches, like Template Model Builder \autocite{osgoodzimmerman2021statistical}.

In spatio-temporal statistics the data we observe are indexed by spatial or temporal location.
The independent and identically distributed (IID) assumptions commonly used for observations are rarely suitable in this setting because we expect there to be spatio-temporal structure.
Often, the latent field is assumed to be jointly multivariate Gaussian.

Latent Gaussian models \autocite{rue2009approximate} are of the form
\begin{align*}
y_i &\sim p(y_i \, | \, \eta_i, \btheta_1), \quad i \in [n]\\
\mu_i &= \mathbb{E}(y_i \, | \, \eta_i) = g(\eta_i), \\
\eta_i &= \beta_0 + \sum_{l = 1}^{p} \beta_j z_{ji} + \sum_{k = 1}^{r} f_k(u_{ki}),
\end{align*}
where \([n] = \{1, \ldots, n\}\).
The response variable is \(\y = (y)_{i \in [n]}\) with likelihood \(p(\y \, | \, \bmeta, \btheta_1) = \prod_{i = 1}^n p(y_i \, | \, \eta_i, \btheta_1)\), where \(\bmeta = (\eta)_{i \in [n]}\).
Each response has conditional mean \(\mu_i\) with inverse link function \(g: \mathbb{R} \to \mathbb{R}\) such that \(\mu_i = g(\eta_i)\).
The vector \(\btheta_1 \in \mathbb{R}^s\), with \(s_1\) assumed small, are additional parameters of the likelihood.
The structured additive predictor \(\eta_i\) may include an intercept \(\beta_0\), linear effects \(\beta_j\) of the covariates \(z_{ji}\), and unknown functions \(f_k(\cdot)\) of the covariates \(u_{ki}\).
The parameters \(\beta_0\), \(\{\beta_j\}\), \(\{f_k(\cdot)\}\) are each assigned Gaussian priors.
It is convenient to collect these parameters into a vector \(\x \in \mathbb{R}^N\) called the latent field such that \(\x \sim \mathcal{N}(0, \bm{Q}(\btheta_2)^{-1})\) where \(\btheta_2 \in \mathbb{R}^{s_2}\) are further parameters, again with \(s_2\) assumed small.
Let \(\btheta = (\btheta_1, \btheta_2) \in \mathbb{R}^s\) with \(m = s_1 + s_2\) be all hyperparameters, with prior \(p(\btheta)\).
Common examples of latent Gaussian models include the following.

Many of the cutting-edge models used in small-area estimation are too complex to be latent Gaussian models.
Examples include disaggregation models, evidence synthesis models \autocite{eaton2019joint,eaton2021naomi}, attendance models, risk group models.
However, many of these models do fit into the class of extended latent Gaussian models \autocite{stringer2021fast}.
By allowing many-to-one link functions, extended latent Gaussian models facilitate modelling of non-linearities.


%%%%% REFERENCES

% JEM: Quote for the top of references (just like a chapter quote if you're using them).  Comment to skip.
% \begin{savequote}[8cm]
% The first kind of intellectual and artistic personality belongs to the hedgehogs, the second to the foxes \dots
%   \qauthor{--- Sir Isaiah Berlin \cite{berlin_hedgehog_2013}}
% \end{savequote}

\setlength{\baselineskip}{0pt} % JEM: Single-space References

{\renewcommand*\MakeUppercase[1]{#1}%
\printbibliography[heading=bibintoc,title={\bibtitle}]}

\end{document}
