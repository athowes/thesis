%%%%%%%%%%%%%%%%%%%%%%%%%%%%%%%%%%%%%%%%%%%%%%%%%%%%%%%%%%%%%%%
%% OXFORD THESIS TEMPLATE

% Use this template to produce a standard thesis that meets the Oxford University requirements for DPhil submission
%
% Originally by Keith A. Gillow (gillow@maths.ox.ac.uk), 1997
% Modified by Sam Evans (sam@samuelevansresearch.org), 2007
% Modified by John McManigle (john@oxfordechoes.com), 2015
% Modified by Ulrik Lyngs (ulrik.lyngs@cs.ox.ac.uk), 2018, for use with R Markdown
%
% Ulrik Lyngs, 25 Nov 2018: Following John McManigle, broad permissions are granted to use, modify, and distribute this software
% as specified in the MIT License included in this distribution's LICENSE file.
%
% John tried to comment this file extensively, so read through it to see how to use the various options.  Remember
% that in LaTeX, any line starting with a % is NOT executed.  Several places below, you have a choice of which line to use
% out of multiple options (eg draft vs final, for PDF vs for binding, etc.)  When you pick one, add a % to the beginning of
% the lines you don't want.


%%%%% CHOOSE PAGE LAYOUT
% The most common choices should be below.  You can also do other things, like replacing "a4paper" with "letterpaper", etc.

% This one will format for two-sided binding (ie left and right pages have mirror margins; blank pages inserted where needed):
%\documentclass[a4paper,twoside]{templates/ociamthesis}
% This one will format for one-sided binding (ie left margin > right margin; no extra blank pages):
%\documentclass[a4paper]{ociamthesis}
% This one will format for PDF output (ie equal margins, no extra blank pages):
%\documentclass[a4paper,nobind]{templates/ociamthesis}
%UL 2 Dec 2018: pass this in from YAML
\documentclass[a4paper, nobind]{templates/ociamthesis}

% UL 5 January 2021 - add packages used by kableExtra
\usepackage{booktabs}
\usepackage{longtable}
\usepackage{array}
\usepackage{multirow}
\usepackage{wrapfig}
\usepackage{colortbl}
\usepackage{pdflscape}
\usepackage{tabu}
\usepackage{threeparttable}
\usepackage{threeparttablex}
\usepackage[normalem]{ulem}
\usepackage{makecell}
\usepackage[colorlinks=false,pdfpagelabels,hidelinks=true]{hyperref}
\usepackage{float}

%UL set section header spacing
\usepackage{titlesec}
% 
\titlespacing\subsubsection{0pt}{24pt plus 4pt minus 2pt}{0pt plus 2pt minus 2pt}

% UL 30 Nov 2018 pandoc puts lists in 'tightlist' command when no space between bullet points in Rmd file
\providecommand{\tightlist}{%
  \setlength{\itemsep}{0pt}\setlength{\parskip}{0pt}}
 
% UL 1 Dec 2018, fix to include code in shaded environments

%UL set whitespace around verbatim environments
\usepackage{etoolbox}
\makeatletter
\preto{\@verbatim}{\topsep=0pt \partopsep=0pt }
\makeatother

%UL 26 Mar 2019, enable strikethrough
\usepackage[normalem]{ulem}

%UL use soul package for correction highlighting
\usepackage{color, soul}
\usepackage{xcolor}
\definecolor{correctioncolor}{HTML}{CCCCFF}
\sethlcolor{correctioncolor}
\newcommand{\ctext}[3][RGB]{%
  \begingroup
  \definecolor{hlcolor}{#1}{#2}\sethlcolor{hlcolor}%
  \hl{#3}%
  \endgroup
}
\soulregister\ref7
\soulregister\cite7
\soulregister\autocite7
\soulregister\textcite7
\soulregister\pageref7

%%%%%%% PAGE HEADERS AND FOOTERS %%%%%%%%%
\usepackage{fancyhdr}
\setlength{\headheight}{15pt}
\fancyhf{} % clear the header and footers
\pagestyle{fancy}
\renewcommand{\chaptermark}[1]{\markboth{\thechapter. #1}{\thechapter. #1}}
\renewcommand{\sectionmark}[1]{\markright{\thesection. #1}} 
\renewcommand{\headrulewidth}{0pt}

\fancyhead[LO]{\emph{\leftmark}} 
\fancyhead[RE]{\emph{\rightmark}} 

% UL page number position 
\fancyfoot[C]{\emph{\thepage}} %regular pages
\fancypagestyle{plain}{\fancyhf{}\fancyfoot[C]{\emph{\thepage}}} %chapter pages

% JEM fix header on cleared pages for openright
\def\cleardoublepage{\clearpage\if@twoside \ifodd\c@page\else
   \hbox{}
   \fancyfoot[C]{}
   \newpage
   \if@twocolumn\hbox{}\newpage
   \fi
   \fancyhead[LO]{\emph{\leftmark}} 
   \fancyhead[RE]{\emph{\rightmark}} 
   \fi\fi}


%%%%% SELECT YOUR DRAFT OPTIONS
% This adds a "DRAFT" footer to every normal page.  (The first page of each chapter is not a "normal" page.)

% This highlights (in blue) corrections marked with (for words) \mccorrect{blah} or (for whole
% paragraphs) \begin{mccorrection} . . . \end{mccorrection}.  This can be useful for sending a PDF of
% your corrected thesis to your examiners for review.  Turn it off, and the blue disappears.
\correctionstrue

%%%%% BIBLIOGRAPHY SETUP
% Note that your bibliography will require some tweaking depending on your department, preferred format, etc.
% If you've not used LaTeX before, I recommend reading a little about biblatex/biber and getting started with it.
% If you're already a LaTeX pro and are used to natbib or something, modify as necessary.
% Either way, you'll have to choose and configure an appropriate bibliography format...


\usepackage[style=authoryear, sorting=nyt, backend=biber, maxcitenames=2, useprefix, doi=true, isbn=false, uniquename=false]{biblatex}
\newcommand*{\bibtitle}{Works Cited}

\addbibresource{references.bib}


% This makes the bibliography left-aligned (not 'justified') and slightly smaller font.
\renewcommand*{\bibfont}{\raggedright\small}


% Uncomment this if you want equation numbers per section (2.3.12), instead of per chapter (2.18):
%\numberwithin{equation}{subsection}


%%%%% THESIS / TITLE PAGE INFORMATION
% Everybody needs to complete the following:
\title{Bayesian spatio-temporal statistics for prioritised HIV prevention}
\author{Adam Howes}
\college{}

% Master's candidates who require the alternate title page (with candidate number and word count)
% must also un-comment and complete the following three lines:

% Uncomment the following line if your degree also includes exams (eg most masters):
%\renewcommand{\submittedtext}{Submitted in partial completion of the}
% Your full degree name.  (But remember that DPhils aren't "in" anything.  They're just DPhils.)
\degree{Doctor of Philosophy}
% Term and year of submission, or date if your board requires (eg most masters)
\degreedate{2023}


%%%%% YOUR OWN PERSONAL MACROS
% This is a good place to dump your own LaTeX macros as they come up.

\newcommand{\Sc}{\mathcal{S}}
\newcommand{\R}{\mathcal{R}}
\newcommand{\N}{\mathcal{N}}
\newcommand{\X}{\mathcal{X}} 
\newcommand{\m}{\mathbf{m}}
\newcommand{\bu}{\mathbf{u}}
\newcommand{\bv}{\mathbf{v}}
\newcommand{\w}{\mathbf{w}}
\newcommand{\x}{\mathbf{x}}
\newcommand{\y}{\mathbf{y}}
\newcommand{\z}{\mathbf{z}}
\newcommand{\bb}{\mathbf{b}}
\newcommand{\bphi}{\bm{\phi}}
\newcommand{\brho}{\bm{\rho}}
\newcommand{\btheta}{\bm{\theta}}

\RequirePackage{bm}

% To make text superscripts shortcuts
	\renewcommand{\th}{\textsuperscript{th}} % ex: I won 4\th place
	\newcommand{\nd}{\textsuperscript{nd}}
	\renewcommand{\st}{\textsuperscript{st}}
	\newcommand{\rd}{\textsuperscript{rd}}

%%%%% THE ACTUAL DOCUMENT STARTS HERE
\begin{document}

%%%%% CHOOSE YOUR LINE SPACING HERE
% This is the official option.  Use it for your submission copy and library copy:
\setlength{\textbaselineskip}{22pt plus2pt}
% This is closer spacing (about 1.5-spaced) that you might prefer for your personal copies:
%\setlength{\textbaselineskip}{18pt plus2pt minus1pt}

% You can set the spacing here for the roman-numbered pages (acknowledgements, table of contents, etc.)
\setlength{\frontmatterbaselineskip}{17pt plus1pt minus1pt}

% UL: You can set the line and paragraph spacing here for the separate abstract page to be handed in to Examination schools
\setlength{\abstractseparatelineskip}{13pt plus1pt minus1pt}
\setlength{\abstractseparateparskip}{0pt plus 1pt}

% UL: You can set the general paragraph spacing here - I've set it to 2pt (was 0) so
% it's less claustrophobic
\setlength{\parskip}{2pt plus 1pt}

%
% Oxford University logo on title page
%
\def\crest{{\includegraphics{templates/ic-small.pdf}}}
\renewcommand{\university}{Imperial College London}
\renewcommand{\submittedtext}{A thesis submitted for the degree of}


% Leave this line alone; it gets things started for the real document.
\setlength{\baselineskip}{\textbaselineskip}


%%%%% CHOOSE YOUR SECTION NUMBERING DEPTH HERE
% You have two choices.  First, how far down are sections numbered?  (Below that, they're named but
% don't get numbers.)  Second, what level of section appears in the table of contents?  These don't have
% to match: you can have numbered sections that don't show up in the ToC, or unnumbered sections that
% do.  Throughout, 0 = chapter; 1 = section; 2 = subsection; 3 = subsubsection, 4 = paragraph...

% The level that gets a number:
\setcounter{secnumdepth}{2}
% The level that shows up in the ToC:
\setcounter{tocdepth}{1}


%%%%% ABSTRACT SEPARATE
% This is used to create the separate, one-page abstract that you are required to hand into the Exam
% Schools.  You can comment it out to generate a PDF for printing or whatnot.

% JEM: Pages are roman numbered from here, though page numbers are invisible until ToC.  This is in
% keeping with most typesetting conventions.
\begin{romanpages}

% Title page is created here
\maketitle

%%%%% DEDICATION -- If you'd like one, un-comment the following.
\begin{dedication}
  For \(\sum_i u_i\)
\end{dedication}

%%%%% ACKNOWLEDGEMENTS -- Nothing to do here except comment out if you don't want it.
\begin{acknowledgements}
 	Thanks to Jeff Eaton and Seth Flaxman for supervision of this research;
 staff and students of the StatML CDT at Imperial and Oxford;
 members of the HIV inference lab at Imperial;
 Mike McLaren, Kevin Esvelt and the Sculpting Evolution lab for hosting my visit to MIT;
 Alex Stringer for hosting my visit to Waterloo;
 the Effective altruism community;
 the Bill \& Melinda Gates Foundation and EPSRC for funding this PhD;
 my friends and family for their support.

 \begin{flushright}
 Adam Howes \\
 Imperial College London\\
 2023
 \end{flushright}
\end{acknowledgements}


%%%%% ABSTRACT -- Nothing to do here except comment out if you don't want it.
\begin{abstract}
	HIV is a large problem.
Disease burden is unevenly distributed.
Effective public health response and prioritised prevention requires accurate, timely, high-resolution estimates of epidemic and demographic indicators.
Complex statistical models are required to overcome significant data challenges.
In this thesis, I develop and apply Bayesian spatio-temporal methods for HIV surveillance.
\end{abstract}

%%%%% MINI TABLES
% This lays the groundwork for per-chapter, mini tables of contents.  Comment the following line
% (and remove \minitoc from the chapter files) if you don't want this.  Un-comment either of the
% next two lines if you want a per-chapter list of figures or tables.
  \dominitoc % include a mini table of contents

% This aligns the bottom of the text of each page.  It generally makes things look better.
\flushbottom

% This is where the whole-document ToC appears:
\tableofcontents

\listoffigures
	\mtcaddchapter
  	% \mtcaddchapter is needed when adding a non-chapter (but chapter-like) entity to avoid confusing minitoc

% Uncomment to generate a list of tables:
\listoftables
  \mtcaddchapter
%%%%% LIST OF ABBREVIATIONS
% This example includes a list of abbreviations.  Look at text/abbreviations.tex to see how that file is
% formatted.  The template can handle any kind of list though, so this might be a good place for a
% glossary, etc.
%tibble(
%  Term = c("HIV"),
%  Abbreviation = C("Human Immunodeficiency Virus")
%  ) %>%
%  arrange(Term) %>%
%  kable(booktab = TRUE, escape = FALSE, "latex")

% After writing, reorder these as they appear in the text

% First parameter can be changed eg to "Glossary" or something.
% Second parameter is the max length of bold terms.
\begin{mclistof}{List of Abbreviations}{3.2cm}

\item[HIV] Human Immunodeficiency Virus.
\item[DALY] Disability Adjusted Life Year.
\item[AIDS] Acquired ImmunoDeficiency Syndrome.
\item[UNAIDS] The Joint United Nations Programme on HIV/AIDS.
\item[PEPFAR] President’s Emergency Plan for AIDS Relief.
\item[Global Fund] Global Fund to Fight AIDS, Tuberculosis, and Malaria.
\item[HIV] Demographic and Health Surveys.
\item[AIS] AIDS Indicator Survey.
\item[PrEP] Pre-Exposure Prophylaxis.
\item[PEP] Post-Exposure Prophylaxis.
\item[FSW] Female Sex Worker(s).
\item[MSM] Men who have Sex with Men.
\item[PWID] People Who Inject Drugs.
\item[TGP] Transgender People
\item[ANC] Antenatal Clinic.
\item[CDC] Centers for Disease Control and Prevention.
\item[UAT] Unlinked Anonymous Testing.
\item[PMTCT] Prevention of Mother-to-Child Transmission.
\item[PLHIV] People Living with HIV.
\item[TaSP] Treatment as Prevention.
\item[STI] Sexually Transmitted Infection.
\item[ART] Antiretroviral Therapy.
\item[VMMC] Voluntary Medical Male Circumcision.
\item[DHS] Demographic and Health Surveys.
\item[PHIA] Population-based HIV Impact Assessment.

\item[MCMC] Markov Chain Monte Carlo.
\item[NUTS] No-U-Turn Sampler.
\item[VI] Variational Inference.
\item[INLA] Integrated Nested Laplace Approximation.
\item[GP] Gaussian Process.
\item[CAR] Conditionally Auto-regressive.
\item[ICAR] Intrisic Conditionally Auto-regressive.
\item[SAE] Small-Area Estimation.
\item[GMRF] Gaussian Markov Random Field.
\item[HMC] Hamiltonian Monte Carlo.
\item[GMRF] Gauss-Markov Random Field.
\item[HMC] Hamiltonian Monte Carlo.
\item[GLM] Generalised Linear Model.
\item[GLMM] Generalised Linear Mixed effects Model.
\item[LGM] Latent Gaussian Model.
\item[ELGM] Extended Latent Gaussian Model.
\item[BF] Bayes Factor.
\item[DIC] Deviance Information Criterion.
\item[BIC] Bayesian Information Criterion.
\item[WAIC] Watanabe-Akaike Information Criterion.
\item[SR] Scoring Rule.
\item[SPSR] Strictly Proper Scoring Rule.
\item[CRPS] Continuous Ranked Probability Score.
\item[ESS] Effective Sample Size.
\item[IID] Independent and Identically Distributed.
\item[PPL] Probabilistic Programming Language.
\item[CCD] Central Composite Design.
\item[EB] Empirical Bayes.

\end{mclistof} 


%%%%% LIST OF NOTATION
\begin{mclistof}{List of Notations}{3.2cm}

\item[$\rho$] HIV prevalence.
\item[$\lambda$] HIV incidence.
\item[$\alpha$] ART coverage.
\item[$\mathcal{S}$] Spatial study region $\mathcal{S} \subseteq \mathbb{R}^2$.
\item[$s \in \mathcal{S}$] Point location.
\item[$\mathcal{T}$] Temporal study period $\mathcal{T} \subseteq \mathbb{R}$.
\item[$t \in \mathcal{T}$] Time. 

\end{mclistof} 

% The Roman pages, like the Roman Empire, must come to its inevitable close.
\end{romanpages}

%%%%% CHAPTERS
% Add or remove any chapters you'd like here, by file name (excluding '.tex'):
\flushbottom

% all your chapters and appendices will appear here
\hypertarget{background}{%
\chapter*{Background}\label{background}}
\addcontentsline{toc}{chapter}{Background}

\adjustmtc
\markboth{Background}{}

\hypertarget{disease-surveillance-and-small-area-estimation}{%
\section{Disease surveillance and small-area estimation}\label{disease-surveillance-and-small-area-estimation}}

\begin{itemize}
\tightlist
\item
  Disease surveillance is a central application of statistics
\item
  Small-area estimation in health, epidemiology and environment
\item
  The Small-Area Health Statistics Unit at Imperial was set-up to monitor health around point sources of environmental pollution in response to the Sellafield enquiry into the increased incidence of childhood leukemia leukaemia near a nuclear reprocessing plant \autocite{elliott1992small}. This research has a focus on ratios of observed events to expected events, and testing hypothesis about hot-spots.
\end{itemize}

\hypertarget{hivaids}{%
\section{HIV/AIDS}\label{hivaids}}

\begin{itemize}
\tightlist
\item
  HIV/AIDS has a large disease burden
\item
  The disease burden is unevenly distributed in space and across communities and individuals
\item
  Surveillance techniques and statistical models have been used to respond to the epidemic
\item
  Key HIV indicators are HIV prevalence, HIV incidence, ART coverage and coverage of other interventions such as PrEP, PEP
\item
  Data difficulties including sparsity in space and time, survey bias, conflicting information sources, hard to reach populations, demography
\item
  Aims for HIV response going forward, and surveillance capabilities are needed to meet them
\item
  Phasing out of nationally-representative household surveys for HIV

  \begin{itemize}
  \tightlist
  \item
    Bayesian survey design
  \end{itemize}
\item
  Importance of relying on multiple sources of information
  Creates requirement for for complex models e.g.~evidence synthesis, Naomi, multivariate models
\item
  Why isn't case-based surveillance included yet?

  \begin{itemize}
  \tightlist
  \item
    There aren't individual linked databases and patient records have to be consolidated
  \item
    Passive case-based surveillance
  \item
    Post-hoc matching and create a case-based surveillance record
  \end{itemize}
\item
  Drivers of transmission
\item
  Possible interventions are ART, condoms, PrEP and PEP, education, economic empowerment, VMMC
\item
  Geographic priorisation versus demographic priorisation: hotspots, key populations, screening and individual level risk characteristics
\item
  Adolescent girls and young women identified as a key demographic, stratification by sexual risk
\item
  Interventions more likely to be demographic specific rather than geographic specific so if majority of difference in effectiveness depends on intervention type then demographic targeting may be more priority
\item
  The population strategy of Geoffrey Rose
\end{itemize}

\hypertarget{bayesian-spatio-temporal-statistics}{%
\section{Bayesian spatio-temporal statistics}\label{bayesian-spatio-temporal-statistics}}

\begin{itemize}
\tightlist
\item
  The practice of doing Bayesian statistics primarily concerns construction of a generative model for the data we observe
\item
  In spatio-temporal statistics, the data is indexed by spatial and or temporal location
\item
  The independent and identically distributed (IID) assumptions commonly used for observations are rarely suitable in the spatio-temporal setting
\item
  We expect there to be spatio-temporal structure
\item
  Given a generative model, computation of the posterior distribution proceeds using approximate Bayesian inference methods
\item
  Markov chain Monte Carlo (MCMC) is the most popular approach and works by simulating samples from a Markov chain which by construction has stationary distribution equal to the distribution of interest
\item
  Variational Bayes approaches assume the posterior distribution belongs to some class and use optimisation to choose the best member of that class
\item
  Laplace approximation and integrated nested Laplace approximation
\item
  Empirical Bayes
\item
  Definition of a latent Gaussian model \autocite{rue2009approximate}
  \begin{alignat}{2}
  &\text{(Observations)}     &        y_i &\sim p(y_i \, | \, x_i, \btheta), \quad i = 1, \ldots, n, \label{eq:data} \\
  &\text{(Latent field)}     &        \x &\sim \mathcal{N}(\x \, | \, \mathbf{0}, \mathbf{Q}(\btheta)^{-1}), \label{eq:process} \\
  &\text{(Parameters)}       & \qquad \btheta &\sim p(\btheta), \label{eq:parameters}
  \end{alignat}

  \begin{itemize}
  \tightlist
  \item
    Common examples
  \end{itemize}
\item
  Examples of models used in HIV inference which are close to being latent Gaussian models, but aren't, and hence can't be fit using INLA

  \begin{itemize}
  \tightlist
  \item
    Disaggregation models
  \item
    Evidence synthesis models like Naomi \autocite{eaton2021naomi,eaton2019joint}
  \item
    Compartmental models
  \item
    ART attendance models
  \item
    Multinomial models like for district-level risk factors

    \begin{itemize}
    \tightlist
    \item
      Multinomial logistic regression
    \end{itemize}
  \end{itemize}
\item
  Other complex models from \href{https://www.bioss.ac.uk/rsse/2013/September2013slides-Illian.pdf}{ecology} that can't currently be fit using INLA
\item
  Definition of extended latent Gaussian models \autocite{stringer2021fast}

  \begin{itemize}
  \tightlist
  \item
    Many-to-one is not an issue for \texttt{R-INLA}, the latent field is implemented as a concatenation of many vectors already. For example, for \(\eta_i = \beta_0 + \phi_i\) with \(i = 1, \ldots, n\) the latent field is \((\eta_1, \ldots, \eta_n, \beta_0, \phi_1, \ldots, \phi_n)^\top\) of dimension \(2n + 1\)
  \item
    For additive models, the only non-linearity is in the link function
  \end{itemize}
\item
  Particular properties of spatio-temporal models (and LGMs) which make INLA, if feasible, often the best option
\item
  The increasing popularity of empirical Bayes approaches, like Template Model Builder \autocite{osgoodzimmerman2021statistical}
\item
  Adaptive Gauss Hermite quadrature (AGHQ), like the central composite design (CCD) and grid strategies, is one way to choose the hyper-parameter integration points in the integrated nested Laplace approximation (INLA)
\item
  Finn Lindgren is working on a method for non-linear predictors, called the \href{https://github.com/inlabru-org/inlabru/blob/55896f10d563c14e34cab577b29b733aac051f86/vignettes/method.Rmd}{iterative INLA method}

  \begin{itemize}
  \tightlist
  \item
    More \href{https://informatique-mia.inrae.fr/reseau-resste/sites/default/files/2020-09/slides-Lindgren_Avignon2018.pdf}{slides} here
  \end{itemize}
\item
  Thesis work of \href{https://cds.cern.ch/record/639625/files/sis-2003-305.pdf}{Follestad} that stayed as a preprint
\item
  How does the ecological fallacy relate to aggregated output models
\end{itemize}

\hypertarget{understanding-models-for-spatial-structure-in-small-area-estimation}{%
\chapter*{Understanding models for spatial structure in small-area estimation}\label{understanding-models-for-spatial-structure-in-small-area-estimation}}
\addcontentsline{toc}{chapter}{Understanding models for spatial structure in small-area estimation}

\adjustmtc
\markboth{Models for spatial structure in small-area estimation}{}

The repository for this work is \href{https://github.com/athowes/areal-comparison}{\texttt{athowes/areal-comparison}}.
Include an edited version of the corresponding paper here.

\hypertarget{spatio-temporal-estimates-of-hiv-risk-group-proportions-for-adolescent-girls-and-young-women-across-13-priority-countries-in-sub-saharan-africa}{%
\chapter*{Spatio-temporal estimates of HIV risk group proportions for adolescent girls and young women across 13 priority countries in sub-Saharan Africa}\label{spatio-temporal-estimates-of-hiv-risk-group-proportions-for-adolescent-girls-and-young-women-across-13-priority-countries-in-sub-saharan-africa}}
\addcontentsline{toc}{chapter}{Spatio-temporal estimates of HIV risk group proportions for adolescent girls and young women across 13 priority countries in sub-Saharan Africa}

\adjustmtc
\markboth{Spatio-temporal estimates of HIV risk group proportions for adolescent girls and young women across 13 priority countries in sub-Saharan Africa}{}

The repository for this work is \href{https://github.com/athowes/multi-agyw}{\texttt{athowes/multi-agyw}}.
Include an edited version of the corresponding paper here.

\hypertarget{simplifying-integrated-nested-laplace-approximation-with-adaptive-gaussian-hermite-quadrature}{%
\chapter*{Simplifying Integrated nested Laplace approximation with adaptive Gaussian Hermite quadrature}\label{simplifying-integrated-nested-laplace-approximation-with-adaptive-gaussian-hermite-quadrature}}
\addcontentsline{toc}{chapter}{Simplifying Integrated nested Laplace approximation with adaptive Gaussian Hermite quadrature}

\adjustmtc
\markboth{Simplifying Integrated nested Laplace approximation with adaptive Gaussian Hermite quadrature}{}

The repository for this work is \href{https://github.com/athowes/elgm-inf}{\texttt{athowes/elgm-inf}}.
Include an edited version of the corresponding paper here.

\startappendices

\hypertarget{the-first-appendix}{%
\chapter{The First Appendix}\label{the-first-appendix}}


%%%%% REFERENCES

% JEM: Quote for the top of references (just like a chapter quote if you're using them).  Comment to skip.
% \begin{savequote}[8cm]
% The first kind of intellectual and artistic personality belongs to the hedgehogs, the second to the foxes \dots
%   \qauthor{--- Sir Isaiah Berlin \cite{berlin_hedgehog_2013}}
% \end{savequote}

\setlength{\baselineskip}{0pt} % JEM: Single-space References

{\renewcommand*\MakeUppercase[1]{#1}%
\printbibliography[heading=bibintoc,title={\bibtitle}]}


\end{document}
